
\chapwithtoc{Conclusion}

In this thesis, we have proposed a design of an experiment in which we might be able to assemble a stable Coulomb crystal with electrons trapped inside. Hopefully, we will cool these electrons to the point when they form a Fermi gas. If successful, such an experiment would advance current options in the research of quantum systems.

We have presented a comprehensive overview of the current state of knowledge regarding the trapping of charged particles solely by an electric field. We have focused on using the ideal Paul trap. We exhibit the supremacy of trapping with two frequencies instead of one for the confinement of two species with widely different charge-to-mass ratios. By reproducing results for trapping in one direction from reviewed scientific articles, we have confirmed the validity of our simulations. After that, we continued producing novel results by studying the stability of electrons in all spatial directions.

We have developed a computer program to simulate charged particles' motion inside the Paul trap. We have used this program to study the stability of particles in dependence on the trap settings, identifying whole areas of stable operating conditions. In the same way, we have evaluated an average electron velocity throughout the trajectory. Based on these results, we propose a technique to achieve lower electron temperatures. To model such a process, we must add a feature to our program representing electrons' possibility to cool themselves by interaction with ions. 

\xxx{This and this was performed and this and this will be extended in the future to assist in determination of optimal experimental conditions for studies of low energy ion-electron interactions. Namely this and this have to done\dots}
