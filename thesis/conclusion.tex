
\chapwithtoc{Conclusion}

In this thesis, we have proposed a design of an experiment offering the possibility of assembling a stable Coulomb crystal with electrons trapped inside. We are aiming to achieve cooling of electrons up to the point when they form a Fermi gas. If successful, such an experiment would expand current lines of research on quantum systems. We have presented a comprehensive overview of the current state of knowledge regarding the trapping of charged particles solely by an electric field, focusing on the ideal Paul trap. We exhibit the supremacy of trapping with two frequencies instead of one for the confinement of two species with widely different charge-to-mass ratios. We have developed a computer program to simulate charged particle motion inside the Paul trap. Using this program, we studied the stability of particles in dependence on the trap settings, identifying whole areas of stable operating conditions. By reproducing results for trapping in one direction from reviewed scientific articles, we have confirmed the validity of our simulations. After that, we continued producing novel results by studying the stability of electrons in all spatial directions. Other than stability, we evaluated an average electron velocity throughout the trajectory dependent on the trap parameters. Based on these results, we propose a technique to achieve lower electron temperatures. To model such a process, we must add a feature to our program representing electrons' possibility to cool themselves by interaction with ions. The program will be further developed for use in the real geometry of our trap. 
