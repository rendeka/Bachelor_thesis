\chapwithtoc{Conclusion}

In this thesis, we have outlined a design of an experiment offering the possibility of assembling a stable Coulomb crystal with electrons trapped inside it. We are aiming to achieve cooling of electrons up to the point when they form a Fermi gas. If successful, such an experiment would expand current lines of research on quantum systems. 

We have presented a comprehensive overview of the current state of knowledge regarding the trapping of charged particles solely by an electric field. We have shown the general treatment of effective potential and stability, applying it to the concrete case of the ideal Paul trap. Based on pseudopotential approximation, we have exhibited the supremacy of trapping with two frequencies instead of one for the confinement of two species with widely different charge-to-mass ratios. We have derived and subsequently studied the stability of equations of motion in all spatial directions in contrast to the reviewed articles, where the stability is discussed only in a single dimension.

We have developed a computer program for simulating charged particles inside the Paul trap. Using this script, we have studied the stability of particles in dependence on the trap settings, successfully identifying whole areas of stable operating conditions. By reproducing results for trapping in one direction from reviewed scientific articles, we have confirmed the validity of our simulations. After that, we continued producing novel results by studying the stability of electrons in all spatial directions. Moreover, we have contributed new unique results by evaluating an average electron velocity throughout the trajectory, depending on the parameters of the trap. We have found that low electron temperatures could be achieved in the weaker electric fields. Based on these results, we propose a technique to achieve successful electron cooling by the consecutive change of trapping parameters. In order to model such a process, we need to add electron cooling caused by interaction with ions to our simulations. We will further study electrons' stability in the presence of CCs. Eventually, the program will be improved to reproduce the results of this thesis in the real geometry of our trap. 
