\chapter{Theoretical introduction}
\label{chap:refs}

In this chapter, we introduce concepts essential for understanding this thesis's concerns. This chapter is divided into several subsections. \footnote{\xxx{In the first subsection, we are going to discuss the basic principles of plasma physics. In the following subsection, we will talk about the motion of a charged particle inside the Paul trap. Next, we will introduce the idea of laser cooling. And finally, in the last subsection, we will present our approach to simulating such a system, entering this thesis's practical component.}} \todo{the sections in the first two chapters will probably be rearranged}

\section{What is plasma?}
\xxx{Things to write about: definition of plasma, characteristic parameters, coupling parameter $\Gamma$, better characterization of Coulomb crystal, point of crystallization, strongly coupled plasma in the nature}

\section{Ion trapping}
Here we introduce the concept of trapping a single ion by a quickly oscillating field. We will tightly follow a classic textbook \cite{gerlich1992inhomogeneous} in the whole section, starting by writing the equation of motion.
\subsection{Equation of motion}
Let us consider a particle with mass $m$, charge $q$ with it's position denoted by vector $\vb{r}$. We insert such a particle into the external time-dependent electromagnetic field described by $\vb{E}(t,\vb{r})$ and $\vb{B}(t,\vb{r})$. The Lorentz force gives the equation of motion:
\begin{equation}
	m \vb{\ddot{r}} = q \left(\vb{E}(t,\vb{r}) + \vb{\dot{r}} \times \vb{B}(t,\vb{r})\right).
\end{equation}
Since we are not using any external magnetic field, and while trapping a particle in a compact space, we usually deal with small velocities. Therefore we can neglect the effect of the term $\vb{\dot{r}} \times \vb{B}$, which means that the equation of motion simplifies to:

\begin{equation}
	\label{equation of motion}
	m \vb{\ddot{r}} = q \vb{E}(t,\vb{r}).
\end{equation}
We further assume that the electric field is composed of static and time-dependent parts. We are looking for a simple periodic time-dependency. A typical way to model such behavior would be $\vb{E}(t) \sim cos(\Omega t)$, giving us:
\begin{equation}
	\vb{E}(t,\vb{r}) = \vb{E_s}(\vb{r}) + \vb{E_0}(\vb{r}) cos(\Omega t).
\end{equation}
	\subsection{Effective potential}
Solving such a differential equation with a rapidly changing right-hand side can be troublesome, albeit not impossible. It will be examined in the section \ref{sec:floquet}. When trapping ions, we are not always interested in exact trajectories. The relevance often lies in the time-averaged effect of a swiftly changing field. With that in mind, we will now try to derive \emph{effective potential} fulfilling precisely this role.
Let's consider initial conditions: $\vb{r}(0) = \vb{r_0}$ and $\vb{\dot{r}}(0) = 0$. For the simplest case of homogeneous electric field $\vb{E_0}(\vb{r}) = const$, we obtain a trivial solution:
\begin{equation}
	\label{solution for homogenious field}
	\vb{r}(t) = \vb{r_0} - \vb{A} cos(\Omega t),
\end{equation}
where the vector: 
\begin{equation}
	\label{oscillation amplitude}
	\vb{A} \equiv \vb{A}(\vb{r}) = \frac{q \vb{E_0}(\vb{r})}{m \Omega^2},
\end{equation}
is an amplitude of oscillation around the initial position of the particle. The crucial consequence of this result is that we can further restrict the motion of a particle by increasing the frequency of field oscillation \footnote{Frequencies used for trapping ions (or even electrons) are in the range of radio frequencies (RF). Therefore, we can treat the electric field in the quasistatic approximation.}. Of course, the situation changes when we bring a small inhomogeneity into the field. Here comes our first leap of fate by assuming that the amplitude of oscillation $\vb{A}$ won't be affected by this inhomogeneity. Instead, the particle will drift slowly towards the weaker field region. Motivated by this observation, we can try to find a solution to the equation of motion in the form:
\begin{equation}
	\label{motion separation}
	\vb{r}(t) = \vb{R_0}(t) + \vb{R_1}(t),
\end{equation}
where $\vb{R_0}(t)$ represents consequence of smooth drift and $\vb{R_1}(t)$ stands for rapid oscillation, expressed as:
\begin{equation}
	\label{oscillation motion}
	\vb{R_1}(t) = - \vb{A} cos(\Omega t).
\end{equation}
If the field amplitude $\vb{E_0}(\vb{r})$
 varies smoothly with regards to the space dimension, we can get by just with its first-order Taylor expansion:
\begin{equation}
	\label{field expansion}
	\vb{E_0}(\vb{R_0}(t) - \vb{A} cos(\Omega t)) \approx \vb{E_0}(\vb{R_0}(t)) -(\vb{A} \cdot \nabla) \vb{E_0}(\vb{R_0}) cos(\Omega t) + \dots.
\end{equation}
Substituting \eqref{motion separation} and \eqref{field expansion} into equation of motion \eqref{equation of motion} \textit{(omitting currently uninteresting static term $\vb{E_s}$)}, we get:
\begin{equation}
	m \big( \vb{\ddot{R}_0(t)} + \vb{\ddot{R}_1(t)} \big) = q \ cos(\Omega t) \big[ \vb{E_0(\vb{R_0}(t))} - (\vb{A} \cdot \nabla) \vb{E_0(\vb{R_0}(t))} cos(\Omega t)  \big].
\end{equation}

Presuming slow spacial variation of vectorfield $\vb{E_0}(\vb{r})$ implies: \\ $|\vb{\ddot{A}}| \ll |\vb{\dot{A}}| \Omega \ll |\vb{A}| \Omega^2$, which we can exploit in time derivative of quickly oscillating term $\vb{R_1}(t)$ \eqref{oscillation motion}, giving us:
\begin{equation}
	\vb{\ddot{R}_1} = - \vb{\ddot{A}} cos(\Omega t) + 2 \Omega \vb{\dot{A}} sin(\Omega t) + \vb{A} \Omega^2 cos(\Omega t) \approx \vb{A} \Omega^2 cos(\Omega t)
\end{equation}

Further substituting for amplitude of oscillation $\vb{A}$ from \eqref{oscillation amplitude} continuing in the spirit of time-averaging:
\begin{equation}
	\vb{A} = \frac{q \vb{E_0}(\vb{r})}{m \Omega^2} \approx \frac{q \vb{E_0}(\vb{R_0}(t))}{m \Omega^2},
\end{equation}
which transfers into $\vb{R_1}$ as:
\begin{equation}
	\label{oscillation motion approx}
	\vb{R_1}(t) = - \frac{q \vb{E_0}(\vb{R_0}(t))}{m \Omega^2} cos(\Omega t). 
\end{equation}
Terms in the equation of motion with dependence on $cos(\Omega t)$ cancel each other out and by using a vector identity:
\begin{equation}
	(\vb{E_0} \cdot \nabla) \vb{E_0} = \frac{1}{2} \nabla E_0^2 - \vb{E_0} \times (\nabla \times \vb{E_0}) = \frac{1}{2} \nabla E_0^2,
\end{equation}
where the second equality follows from Maxwell equation for quasistatic field: \\ $\nabla \times \vb{E_0} = 0$. By replacing term $cos^2(\Omega t)$ with its mean value $\overline{cos^2(\Omega t)} = \nicefrac{1}{2}$ we finally obtain:
\begin{equation}
	m \vb{\ddot{R}_0} = \frac{q^2}{4 m \Omega^2} \nabla E_0^2.
\end{equation}
Now by resurrecting the static field term as $\vb{E_s} = - \nabla \Phi_s$, we can define the effective potential:
\begin{equation}
	\label{effective potential}
	V^*(\vb{R_0}) = \frac{q^2 E_0^2(\vb{R_0})}{4 m \Omega^2} + q \Phi_s, 
\end{equation}
describing the time-averaged force on a charged particle:
\begin{equation}
	\label{effective equation of motion}
	m \vb{\ddot{R}_0} = - \nabla V^*(\vb{R_0}). 
\end{equation}
This equation is much easier to solve and study than the original equation of motion \eqref{equation of motion} as it does not involve any explicit time-dependency. After solving it, we can quickly obtain the term $\vb{R_1}(t)$ from \eqref{oscillation motion approx} and get an approximative solution to the original equation of motion. From the Fourier analyses of numerically exact solutions \cite{gerlich1992inhomogeneous} we know about the presence of higher-order terms: $$\vb{r}(t) = \vb{R_0}(t) + \vb{R_1}(t) + \vb{R_2}(t) + \dots$$, where $\vb{R_2}(t) + \dots$ are referred to as micro oscillations. We must be careful about keeping the space variation of $\vb{E_0}(\vb{r})$ sufficiently small. Otherwise, these micro oscillations can become large enough to disturb the trajectory of a particle significantly.

\subsection{Adiabacity}

\xxx{\dots}

\subsection{Mathieu equation}
\xxx{should cite \cite{trypogeorgos2016cotrapping} as well}
The equation similar to our equation of motion \eqref{equation of motion} was first excessively studied by Mathieu in 1868. The secular frequency is proportional to the driving frequency as:
\begin{equation}
	\omega \approx \frac{\Omega}{2} \sqrt{a + \frac{q_1^2}{2}}
\end{equation}

\subsection{Real geometry of our trap}
\todo{need to add pictures throughout the whole thesis}
Since we want to implement laser cooling in our experiment, the linear Paul trap is not a viable option, as its apparatus would stand in the way of laser beams. For this reason, we will use surface electrodes where the particles levitate above the trap so that the ions will be accessible to us. Nevertheless, we will start our research by examining the situation with the geometry of the ideal linear Paul trap.


\subsection{Linear multipole Paul trap}
\xxx{continue with \cite{gerlich1992inhomogeneous}}

\subsection{Stability}
\xxx{The concept of stability is not straightforward to define for our system. We have Lagrange stability, Liapunov stability\dots First, we want to be working within the condition for adiabaticity so that the dynamic field will not augment the particle's energy. Inside this restriction, we will further search for characterization of stability.}

The linear Paul trap has its characteristic length $\mathcal{R}_0$ defined by the distance from the middle of the trap to the electrode. We will use a criterion of stability established in \cite{gerlich1992inhomogeneous} declaring a solution stable if it satisfies the condition: \\ $\max\limits_{x \in \mathcal{L}}(r) \leq 0.8 \ \mathcal{R}_0$, where $\mathcal{L}$ is the whole trajectory of a particle, is declared stable. The drawback of this definition is that we must keep the simulation going long enough to account for the slowly diverging particles.

\section{Laser cooling} 

The Ca+ ion has an energy gap between the ground and one of its excited states with the value corresponding to the wavelength of \SI{397}{\nano\meter} \cite{urabe1993laser}. By tuning the wavelength of our laser slightly below this transition energy, we can exploit the Doppler effect so that only ions moving towards the laser can experience radiation with the right frequency to excite them. After a brief time, the atom will deexcite, emitting a photon in a random direction. The only way the ion would still have the same momentum as before the absorption is if the photon was emitted exactly in the same direction \todo{rephrase this sentence} as it was absorbed (as if the photon did not interact with the atom at all). But since the photon emission is isotropic, the ion will effectively slow down. This type of laser cooling is also known as \emph{Doppler cooling}. More detailed explanation can be found in \cite{alma990008711500106986}.  
 
	
\section{Simulation}
\label{simulation}

Let us begin this section by summarizing our approximations (inspired by \cite{Friedman_1982}) already used when deriving the equation of motion $\rightarrow$ starting with insignificant ones. 

\begin{description}
	\item[Gravitational interaction:] neglecting gravitational interaction goes without saying since, for Ca+ ions, it is weaker than electrostatic force by order of $\sim 10^{32}$.
	\item[Induced charge on the electrodes:] charged particles will induce surface charge density on the electrodes made from electrically conductive material. This causes attraction of a particle toward the electrode, which can contribute to vacation of the particle from the trap. We will neglect this effect since our definition of stable trajectory does not allow the particle to approach to the electrode close enough for this effect to by significant.
	\item[Relativistic effects:] we did not involve any relativistic corrections since while trapping particles we are dealing with small velocities.
	\item[Ion radiation:] well known consequence of Maxwell equations is that accelerating charged particle emits electromagnetic radiation. We did not account for this energetic loss.
	\item[Magnetic field:] created by presence of quickly changing electric field and moving charges is neglected since we work with weak fields and slow particles. 
\end{description}	
Here ends the list of phenomena whose oversight should not have any significant effect on our results. It could be the case that to get realistic results we might need to account for some of the following effects.
\begin{description}
	\item[Collisions with neutrals:] \xxx{we will be able to make a vacuum with pressure of (no idea and not sure it it can produce significant problems)} 
	\item[Phase shift:] induced by the finite speed of electrons inside an electrode and the final speed of light since we will use quite a small trap and high frequencies. \xxx{not sure if it can bring significant complications} \todo{if I have understood from today's meeting correctly, then we handle this problem by dividing our electrodes into eight sections, each with its own feeding.}
\end{description}

\subsection{The code}
The practical part of this thesis consists of developing the code simulating the motion of ions and electrons in two frequency Paul trap. We have chosen the programming language python for its current popularity allied with an abundance of highly optimized libraries and a good combination of computational and development costs. The source code \xxx{(will probably be free to use)} can be found at \href{https://github.com/rendeka/Bachelor_thesis.git}{github\footnote{$https://github.com/rendeka/Bachelor\_thesis.git$}.} \xxx{Main features of the program are:
\begin{itemize}
	\item Creating a Coulomb crystal.
	\item Making stability diagram in dependence on $q_1$ and $q_2$ parameters.
	\item Parallelizing the computation of stability diagram. 
	\item Optimizing the algorithm to compute stability only on the edge of stability regions.
	\item Tracking the information about the system: positions, velocities, energies
\end{itemize}
I think it would be beneficial to explain, for example what I mean by "optimizing," but I do not think it belongs here. I will probably put it into the appendix.
} 