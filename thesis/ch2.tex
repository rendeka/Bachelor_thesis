\chapter{Co-trapping of two different species}
\label{chap:co-trapping}

\section{Two frequency quadrupole Paul trap}
\label{sec:two frequency trap}
As was mentioned in the section \ref{sec:spring constant},
we can use two frequencies, each targeting to optimize the trapping of one species. Higher frequency $\Omega_2$ for trapping electrons, with the potential $V_2$ applied to the electrode. A lower frequency $\Omega_1$ with potential $V_1$ for trapping ions. We have already derived the ideal quadrupole potential for a single frequency case. Repeating the same process with the exception of using applied potential on the electrodes in the form $\Phi_0 = V_0 + V_1 \cos(\Omega_1 t) + V_2 \cos(\Omega_2 t)$, would yield two-frequency potential for an ideal quadrupole:
\begin{equation}
	\label{ideal quadrupole potential two frequency}
	V(t,\vb{r}) = \left[V_0 + V_1 \cos(\Omega_1 t) + V_2 \cos(\Omega_2 t)\right] \dfrac{x^2+y^2 \minus 2z^2}{2\ell_0^2}.
\end{equation}
The equations of motion for a particle in such potential, after a change of variable: $\tau = \nicefrac{t\Omega_2}{2}$ are:
\begin{subequations}
\label{eq of motion for simulation}
\begin{align}
	\ddot{x}(\tau) & = x(\tau) \bigg[ a \minus 2 q_1 \cos\left(2 \tau \nicefrac{\Omega_1}{\Omega_2} \right) \minus 2 q_2 \cos(2\tau) \bigg], \\
	\ddot{y}(\tau) & = y(\tau) \bigg[ a \minus 2 q_1 \cos\left(2 \tau \nicefrac{\Omega_1}{\Omega_2} \right) \minus 2 q_2 \cos(2\tau) \bigg], \\
	\label{studied eq of motion}
	\ddot{z}(\tau) & = \minus 2z(\tau) \bigg[ a \minus 2 q_1 \cos\left(2 \tau \nicefrac{\Omega_1}{\Omega_2} \right) \minus 2 q_2 \cos(2\tau) \bigg],
\end{align}
\end{subequations}
where $a$, $q_1$ and $q_2$ are dimensionless parameters:
\begin{subequations}
\label{Mathieu params}
\begin{align}
	\label{a}
	a & = 4 \dfrac{Q V_0}{M\Omega_2^2 \ell_0^2}, \\
	\label{q_1}
	q_1 & = \minus 2 \dfrac{Q V_1}{M\Omega_2^2 \ell_0^2}, \\
	\label{q_2}
	q_2 & = \minus 2 \dfrac{Q V_2}{M\Omega_2^2 \ell_0^2}.
\end{align}
\end{subequations}
Although we will not use any static potential in our case, we keep the term $\sim a$ for the generality of our equations.

Sometimes, instead of solving exact equations of motion, we can get along by simulating a charged particle in an effective potential, significantly reducing computational time. We will use this option for creating a Coulomb crystal. When trapping electrons, the condition \eqref{frequency inequality} must be satisfied; hence we can use already derived single frequency effective potential \eqref{effective potential hyperbolic}:
\begin{equation}
	\label{electron pseudopotential}
	V_{el}^*(\vb{r}) = \frac{M_{el} \Omega_2^2 q_2^2}{16}(x^2 + y^2 + 4 z^2),
\end{equation}
where now, we have set the static field to zero and substituted for $q_2$ from \eqref{q_2}. In pseudopotential for ions, both frequencies will play their parts. Let us remark that Mathieu parameters \eqref{Mathieu params} differ for the case of electron and ion by a factor of $\nicefrac{M_{el}}{M_{ion}}$. Using the Mathieu parameters evaluated for electrons, the pseudopotential for ions \cite{leefer2017investigation} is:
\begin{equation}
	\label{ion pseudopotential}
	V_{ion}^*(\vb{r}) = \frac{M_{el}^2 \Omega_2^2}{16 M_{ion}} \left( \left(\frac{\Omega_2}{\Omega_1}\right)^2 q_1^2 + q_2^2 \right) \left(x^2 + y^2 + 4z^2\right).
\end{equation} 

We proceed by setting up a two-frequency trap as described in \cite{FOOT2018117, trypogeorgos2016cotrapping}. First, we must acknowledge that while the higher frequency can only improve the stability of heavier ions, the slower field can easily misguide the electron. Such a latent instability is commonly anointed as the \emph{parametric excitation}. We have to ensure that the trapping of electrons due to the higher frequency field is strong enough to withstand this undesirable effect. We can secure that to some extent by choosing potentials on electrodes in such a way that $V_1 \ll V_2$. We are able to provide potentials up to $V_2 \sim \SI{100}{\volt}$, and $V_1 \sim \SI{5}{\volt}$, in the conditions of our experiment. Let us now look at each frequency setup independently. Using \eqref{spring constant}, we get a ratio of spring constants:
\begin{equation}
	\label{two frequency ratio}
	\Kappa = \dfrac{\kappa_{el}(\Omega_2)}{\kappa_{ion}(\Omega_1)} \approx \dfrac{M_{ion}}{M_{el}} \left(\dfrac{V_2}{V_1}\dfrac{\Omega_1}{\Omega_2}\right)^2.
\end{equation}
The correct way to create a stable configuration of a two-frequency trap is to begin by finding the optimal parameters for the confinement of the lighter species. Optimal trapping in a single frequency trap corresponds to \cite{gerlich1992inhomogeneous} $q_2 \approx 0.4$. Using \eqref{q_2}, with the choice of $V_2 = \SI{100}{\volt}$, we obtain optimal frequency for electron trapping as $\Omega_2 = \SI{1.88e10}{rad.s^{-1}}$. Now, we must choose the $\Omega_1$ and $V_1$ for ion trapping in a way that suppresses parametric heating of electrons. It is crucial to keep the electron's secular frequency higher than the driving frequency for ion confinement. Otherwise, the $\Omega_2$ frequency field would be too slow to save an electron from parametric excitation. So another condition that must be satisfied in two frequency trapping is:
\begin{equation}
	\label{frequency inequality}
	\Omega_1 \ll \omega_2 = \dfrac{Q V_2}{\sqrt{2} M_{el} \ell_0^2 \Omega_2} \approx \SI{2.65e9}{rad.s^{-1}},
\end{equation}
where the second equality follows from \eqref{secular frequency}. Unfortunately \cite{FOOT2018117}, it is not possible to forestall parametric heating all the time. In the case of $$\mu \Omega_1 = 2 \omega_2,$$
where $\mu \in \N$, we can expect elections' instability due to resonance of driving $\Omega_1$ and secular $\omega_2$ frequency. This estimation clarifies the origin of unstable tongues in stability diagrams; see \ref{chap:results} the last chapter.


We aim to create a Coulomb crystal and trap electrons within so that a cooled electron could spread its de Broglie wavelength across multiple ions. From \eqref{two frequency ratio} we see that it is possible to achieve plausible spring ratios by making the fraction $\nicefrac{\Omega_1}{\Omega_2}$ small. However, we must not forget that lowering the frequency $\Omega_1$ causes weaker confinement of both species. Intending to find a suitable middle ground, we chose $\Omega_1$ such that $\nicefrac{\Omega_2}{\Omega_1} \approx 833$ and the corresponding spring constant ratio is $\Kappa \approx 42$.

\section{Floquet theory}
\label{sec:floquet theory}
We are lucky to have a theory covering linear first-order ODEs with periodic coefficients, meaning the equations in the form:
\begin{equation}
	\label{Floquet structure}
	\vb{\dot{u}}(\tau) = \tensorq{T}(\tau) \vb{u}(\tau),
\end{equation}
where $\tensorq{T}$ is a matrix valued function with minimal period $T$. Let's illustrate this theory for the case of our differential equation. We begin by rewriting the equation \eqref{eq of motion for simulation} as one general system of two first-order differential equations written in the matrix form:
\begin{align}
\label{matrix mathieu}
	\frac{d}{d \tau}
	\begin{bmatrix}
		\zeta(\tau) \\
		\dot{\zeta}(\tau)
	\end{bmatrix}	
	&=
	\begin{bmatrix}
		0 & 1 \\
		\tilde{a} \minus 2 \tilde{q}_1 \cos\left(2 \tau \nicefrac{\Omega_1}{\Omega_2} \right) \minus 2 \tilde{q}_2 \cos(2\tau) & 0	
	\end{bmatrix}
	\begin{bmatrix}
		\zeta(\tau) \\
		\dot{\zeta}(\tau)
	\end{bmatrix},
\end{align}
where:
\begin{align}
	\label{xi variable}
	\zeta &\in \{x,y,z\}, \\
	\label{tilde params} 
	\{ \tilde{a}, \tilde{q}_1, \tilde{q}_2 \} &= \begin{cases}
		\{a, q_1, q_2\} &if \ (\zeta = x) \lor (\zeta = y), \\
		\minus 2 \{a, q_1, q_2\} &if \ (\zeta = z). 
	\end{cases}
\end{align}
This system \eqref{matrix mathieu} already has a structure of \eqref{Floquet structure}. Without the necessity of finding a solution to this system, we can acquire knowledge about its stability. The information we are interested in is whether a solution is bounded for a given set of parameters or not. In this section, we will limit ourselves to the driving frequencies, which can be represented as: $\nicefrac{\Omega_2}{\Omega_1} \equiv \nicefrac{m}{n}$, where $m$ and $n$ are integers and $\nicefrac{m}{n}$ is an irreducible fraction. Then the matrix in equation \eqref{matrix mathieu} is $T = m \pi$ periodic. As in \cite{leefer2017investigation}, we identify the edge of stability regions as a set of parameters for which a solution of \eqref{matrix mathieu} is a $T$ or $2T$ periodic function\footnote{Note that such stability condition differs from the one we demand while simulating the motion of a particle. The periodic solution of equation of motion implies boundedness, but the value of this boundary might lie outside the physical dimensions of the trap. That is why we ought not to be surprised when we find some differences in stability diagrams, even for the simplest case of a single particle in the trap. \label{foot:different stability condition}}. This allows us to seek a solution to our problem in the form:
\begin{equation}
	\label{Floquet ansatz}
	\zeta(\tau) = \sum_{k=-\infty}^{\infty} c_k \exp(i\frac{k}{m}\tau),
\end{equation}
where $c_k$ are constant coefficients. Substituting this into equation \eqref{matrix mathieu} yields an identity:

\begin{multline}
	\sum_{k=\minus\infty}^{\infty}\left[ \left(\tilde{a} \minus \dfrac{k^2}{m^2} \right)c_k \minus \tilde{q}_1 \left(c_{k-2n} + c_{k+2n} \right) \minus \tilde{q}_2 \left(c_{k-2m} + c_{k+2m} \right)  \right] \\ \exp(i\frac{k}{m}\tau) = 0,
\end{multline}
which holds for every $\tau$ only if each element of the sum is equal to zero. This relation can be written as:

\begin{equation}
	\label{Floquet edge eq}
	\tensorq{F} \cdot \begin{bmatrix}
	\vdots \\
	c_{k-1} \\
	c_{k} \\
	c_{k+1} \\
	\vdots
	\end{bmatrix} = \vb{0},
\end{equation}
where $\tensorq{F}$ is an infinite matrix with elements:

\begin{equation}
	\tensorq{F}_{i j} = \left[ \left(\tilde{a} \minus \dfrac{k^2}{m^2} \right)\delta_{i j} \minus \tilde{q}_1 \left(\delta_{i j-2n} + \delta_{i j+2n} \right) \minus \tilde{q}_2 \left(\delta_{i j-2m} + \delta_{i j+2m} \right)  \right],
\end{equation} 
where:
\begin{equation}
	\delta_{ij} = 
	\begin{cases}
		1 & i=j, \\
		0 & i \neq j.
	\end{cases}	
\end{equation}
The equation \eqref{Floquet edge eq} is equivalent to:
\begin{equation}
	\label{Floquet determinant}
	det(\tensorq{F}) = 0.
\end{equation}
So the determination of stability boils down to computing a determinant of a matrix $\tensorq{F}$. We approximate $\tensorq{F}$ by sufficiently large $\rightarrow (10m+1) \times (10m+1)$ finite matrix. For the index $k$ in previous equations it means: $$k \in \{ \minus 5m, \ \minus 5m+1, \  \cdots, \ 5m \minus 1, \ 5m \} \subset \Z,$$ neglecting solutions with smaller periods. Parameters for which $det(\tensorq{F}) > 0$ were identified as stable and for $det(\tensorq{F}) < 0$ as unstable. Since the particle experiences the most instabilities in the z-direction, we focus on the case $\zeta = z$ in the determinant solutions in the last chapter \ref{chap:results}.
 	
\section{Simulation}
\label{sec:simulation}
Let us begin this section by summarizing our approximations, some of which we have already applied while deriving the equation of motion. We follow mainly the overview from \cite{Friedman_1982}. Starting with insignificant neglections and moving towards more problematic ones. 
\begin{description}
	\item[Gravitational interaction:] neglecting gravitational interaction goes without saying since, for Ca+ ions, it is weaker than electrostatic force by order of $\sim 10^{32}$.
	\item[Relativistic effects:] we did not involve any relativistic corrections since we usually deal with small velocities while trapping particles. Ca+ ions are Doppler cooled down to energies of $\sim \SI{e-4}{\eV}$. The fastest simulated electrons had a kinetic energy of $\sim \SI{1}{\eV}$, for which the relativistic gamma factor is still $\gamma \approx  1$ up to the fifth decimal place.
	\item[Ion radiation:] a well-known consequence of Maxwell equations is that accelerating charged particle emits electromagnetic radiation. The power $[P] = [Watt]$ of such radiation can be for our non-relativistic case calculated using \cite{larmor1897lxiii} the Larmor formula: $$P = \dfrac{Q^2 a_c^2}{6 \pi \varepsilon_0 c^3},$$ where $c$ is a speed of light, $\vb{a_c}$ denotes the acceleration of a particle at the given time, and $\varepsilon_0$ is the vacuum permittivity. We can estimate this radiation power for electron by taking an average\footnote{Average value defined as: $\langle \xi \rangle = \nicefrac{1}{T} \int_{0}^{T} \xi(\tau) \, d\tau,$ where $T$ denotes a total time of simulation.} value of it's acceleration during stable trajectories from our simulations: $\langle a_c \rangle \sim \SI{1e17}{m.s^{-2}}$. The resulting radiation power is then: $\langle P \rangle \approx \SI{5e-20}{\watt}$ which has, in the timescale of our simulations $\sim \SI{e-7}{s}$ a negligible effect. Radiation power for ions would be even less significant.
%$\varepsilon_0 = \SI{8.85e-12}{m^{-3}.kg^{-1}.s^{4}.A^{2}}$
	\item[Electromagnetic field:] Let us begin by writing full form Maxwell's equations assuming we can provide sufficient vacuum to disregard material properties:
\begin{subequations}
\label{full maxwell}
\begin{align}
	\label{full gauss}
	&\nabla \cdot \vb{E} = \dfrac{\rho}{\varepsilon_0}, \\
	\label{full faraday}
	&\nabla \times \vb{E} = \minus\dfrac{\partial\vb{B}}{\partial t}, \\
	\label{full magnet}
	&\nabla \cdot \vb{B} = 0, \\
	\label{full ampere}
	&\nabla \times \vb{B} = \dfrac{1}{c^2} \left( \dfrac{\vb{j}}{\varepsilon_0} +  \dfrac{\partial\vb{E}}{\partial t} \right),
\end{align}
\end{subequations}
where $\rho$ is a charge density and $\vb{j}$ is current density. We are using quasistatic approximation instead:
\begin{subequations}
\label{static maxwell}
\begin{align}
	\label{static gauss}
	&\nabla \cdot \vb{E} = 0, \\
	\label{static faraday}
	&\nabla \times \vb{E} = \vb{0}, \\
	\label{static magnet}
	&\nabla \cdot \vb{B} = 0, \\
	\label{static ampere}
	&\nabla \times \vb{B} = \vb{0}.
\end{align}
\end{subequations}
We are setting $\rho=0$ because we derive our equations in charge-free space. The Coulomb interaction is included in the simulation afterwards, exploiting the linearity of Maxwell's equations. The current density inside our trap caused by a moving electron with maximal speed throughout our simulations $v_{max} \sim \num{e+4}$, can be estimated as:
\begin{equation}
	j = \rho v \approx \nicefrac{Q v}{\ell_0^3} \sim \SI{e-4}{A.m^{-2}}.
\end{equation}
Overall, the amplitude of an oscillating magnetic field is much smaller than that of an electric \cite{Friedman_1982}: $\nicefrac{E_0}{B_0} = c$. This means that for a magnetic field to play any significant role in the equation of motion, the particle would have to have a speed comparable to the speed of light, which is not the case, as we have already discussed. Even though we are not using any external magnetic field, the validity of equation \eqref{static faraday} requires one more condition \cite{Friedman_1982} for a wavelength of our changing electric field, and that is:
\begin{equation}
	\lambda \equiv \dfrac{2\pi c}{\Omega_2} \gg \ell_0,
\end{equation}
after substituting corresponding values we get: $\lambda = \SI{100}{\mm} \gg \SI{0.5}{\mm} = \ell_0$. If this condition was not met, there could be a possibility of forming standing waves that could change the trap's dynamics.  
	\item[Induced charge on the electrodes:] charged particles induce surface charge density on the electrodes. This causes attraction of a particle toward the electrode, which can contribute to vacation of the particle from the trap. For a reference, we can look at simplified situation of a charged particle near an infinite conductive plane. As we have already stated, an average acceleration of the electron throughout the stable simulated trajectory was approximately $\langle a_c \rangle \sim \SI{e17}{m.s^{-2}}$. For an infinite plane to cause an acceleration $\sim \nicefrac{\langle a_c \rangle}{100}$, the electron would have to approach the electrode up to the distance of $\approx \SI{0.25}{\micro\meter}$, which would already be outside our definition of stable trajectory. Hence we can omit such an effect. 
	\item[Creation of Rydberg atoms:] an electron can get caught in some highly excited ion orbital creating a Rydberg atom. The electron can subsequently drop to a lower but still volatile state and vacate from it under the influence of the trap, losing energy by the associated photon emission. An electron moving inside a Coulomb crystal can be within reach of multiple ions simultaneously. Hence it may greatly contribute to electron cooling. It will be necessary to add such a process to our simulation in the future.
	\item[Phase shift:] is caused by the finite speed of electric signal delivered to the electrode. This could be problematic since the characteristic dimension of our trap is $\ell_0 = \SI{0.5}{\mm}$, and we are using frequencies up to the orders of $\Omega_2 \sim \SI{e10}{rad.s^{-1}}$. We tackle this problem by dividing our electrodes into eight sections, see \ref{fig:Real trap geometry 1}, each with its own feeding from the power source.
	\item[Imperfection of electrodes:] adding some perturbation to the electric potential caused by the imperfection of electrodes and the following study of its impact on stability will be one of the subjects for our future research.	
	\item[Collisions with neutrals:] we will be able to make a vacuum with pressure of $P\sim \SI{e-13}{bar}$. At this pressure, the heaviest remaining gas component is helium atoms. The collision frequency $z$ is defined as:
\begin{equation}
	\label{collision frequency}
	z = \frac{Z}{V} = \frac{N_{el} N_{He}}{V} \sigma \sqrt{\frac{8 k_b T}{\pi \mu}},
\end{equation}
where $N_{el}$ and $N_{He}$ are electron and helium counts respectively, $\mu$ is a reduced mass, $\sigma$ is the collision cross-section, and $V$ is a volume of the gas. The reduced mass for an electron and a helium atom is:
\begin{equation}
	\label{reduced mass}
	\mu = \frac{M_{el} M_{He}}{M_{el} + M_{He}}.
\end{equation}
Looking at helium as an ideal gas in thermodynamic equilibrium, we can derive the number of helium atoms as:
\begin{equation}
	\label{number of helium atoms}
	N_{He} = \frac{P V}{k_b T}.
\end{equation}
The number of electrons will be in the orders of $N_{el} \sim 1$. We can approximate electron-helium collision as a collision of hard spheres. In that case, the cross-section is equal to $\sigma \approx \pi r_w^2$, where $r_w$ is a van der Waals radius\footnote{In the hard sphere model, van der Waals radius indicates a radius of an atom.} of a helium atom: $r_w = \SI{2.31e-10}{m}$. Assuming room temperature $T=\SI{300}{\kelvin}$ helium gas, we get a final collision frequency:
\begin{equation}
	\label{collision frequency value}
	z = r_w^2 P \sqrt{\frac{8\pi}{\mu k_b T}} \approx \SI{4.36e-2}{\hertz},
\end{equation}
which means that we could observe such collisions in our experiment. However, the occurrence of these collisions is very unlikely in the time scales of our simulations; hence we do not include them. Nevertheless, as we will show, these collisions could affect long-term electron stability. It is not hard to prove that in the case of solid spheres, the change in a helium's kinetic energy after one elastic collision with an electron is\todo{not $100\%$ sure whether this is correct}:
\begin{equation}
	\label{change in energy after collision}
	\frac{\Delta E_k^{He}}{E_k^{He}} = \minus 2\frac{M_{el} M_{He}}{(M_{el} + M_{He})^2} \left(1 \minus \cos \chi \right) = \minus\frac{\Delta E_k^{el}}{E_k^{He}},
\end{equation}
where $\chi$ is a dispersion angle. The maximal change in kinetic energy happens for $\chi = \pi$. Considering helium atoms at a room temperature $T = \SI{300}{\kelvin}$, the value of this energy change is $\Delta E = \SI{0.16}{\kelvin}$, which would be enough to kick a weakly confined electron out of the trap\footnote{In our experiment, we will be able to create a potential well for ions with a maximum depth of $\SI{32}{\milli\kelvin}$, and for electrons with maximal depth up to $\SI{15}{\kelvin}$.}.
\end{description}	

\subsection{Choice of time-step}
When looking at the electrons' stability, we use a total simulation length corresponding to twenty secular oscillations $(\nicefrac{20}{\omega})$ \eqref{secular frequency}, if we do not say otherwise. When examining the average velocity of an electron, it was sometimes necessary to extend this time up to two-hundred secular oscillations to harvest the practically relevant information. Choice of time-step is always a delicate issue. We used a constant time-step to solve differential equations with the explicit time dependency, following the Nyquist criterion \cite{1697831}. Nyquist criterion is used mainly in processing compact signals with convergent Fourier series. It states that for the signal with the highest frequency $\textit{f} \equiv \nicefrac{\Omega}{2\pi}$, the largest possible sample size so that the discretization of the signal will carry equivalent information as the continuous one is $\nicefrac{1}{(2\textit{f})}$. We started numerically solving the equation of motion with this time step, gradually decreasing its size until the numerical solution with the denser sampling would give us the same result for the given tolerance. Repeating this process for several numerical methods \cite{teukolsky1992numerical} for integrating ODEs. For the case of a single electron in the trap had the best performance a predictor-corrector method. The sufficient time-step was $\Delta t = \nicefrac{1}{(10 \Omega)}$. For simulations including multiple particles, we have used a Verlet method with an adaptive time-step.

\subsection{Treatment of laser cooling}
In our simulation, we treated the laser cooling of ions by introducing a frictional force\footnote{Meaning a force proportional to $\propto \minus \vb{\dot{r}}$.}.  The strength of this force is characterized by the parameter $\beta \in (0,1)$. 

\subsection{Simulating Coulomb crystal}
We simulated Coulomb crystal by molecular dynamics, meaning we solved the equation of motion for each ion in effective potential, including Coulomb interactions and damping. The damping parameter $\beta$ was chosen large enough to decelerate the particles within a computationally reasonable time. To ensure that the ions had enough time to find a potential minimum, they were given a synthetic boost in kinetic energy every time their temperature reached below $\SI{0.01}{Kelvin}$. 

\subsection{The code}
The practical part of this thesis consists of developing the code simulating the motion of ions and electrons in a two frequency Paul trap. We have chosen the programming language python for its current popularity allied with an abundance of highly optimized libraries, and a good combination of computational and development costs. The source code can be found at \href{https://github.com/rendeka/Bachelor_thesis.git}{github\footnote{\href{https://github.com/rendeka/Bachelor_thesis.git}{$https://github.com/rendeka/Bachelor\_thesis.git$}}.} Main features of the program are:
\begin{itemize}
	\item Creating a Coulomb crystal.
	\item Making stability diagram in dependence on $q_1$ and $q_2$ parameters.
	\item Parallelizing the computation of stability diagram. 
	\item Optimizing the algorithm to compute stability only on the edge of stability regions.
	\item Tracking the information about the system: positions, velocities, energies
	\item Producing graphical outcomes.
\end{itemize}
More about its functionalities can be found in the appendix \ref{sec:code}.