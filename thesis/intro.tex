\chapwithtoc{Introduction}

This thesis' practical aim is to contribute to developing an experiment initiated by my supervisor Mgr. Michal Hejduk, Ph.D. In this experiment, we wish to create and study the properties of a quite unusual type of matter, the Coulomb crystal (CC). CCs are mostly stationary structures of ions characterized by large coupling parameter $\Gamma$\footnote{All the non-trivia will be further explained in the first chapter}. These structures have been extensively studied now for decades. Our ambition is to introduce electrons to such a crystal and cool everything down. We will aim for sub-kelvin temperatures when electrons' de Broglie wavelength is greater than the distance between them, forming so-called Fermi gas, which has not been achieved inside a CC up to the current date. Creating a CC means confining a certain number of charged particles in bounded space. The first thing standing in our way is Earnshaw's theorem \cite{earnshaw1848nature}, stating that there is no stable electrostatic configuration of charged particles. Not feeling like giving up, we must try our chances outside the realm of electrostatics. Here we are already presented with two well-established ways of storing charged particles. One utilizes an axial magnetic field to confine particles in a radial direction and a static electric field for confinement in an axial direction. This approach developed by H.G. Dehmelt is called the Penning trap. The second option to restrict the movement of charged particles in all directions is to use the dynamic and static electric fields solely. The Pioneer of this technique was Wolfgang Paul. Both these gentlemen were awarded a shared Nobel prize for physics in 1989 for their efforts in this field. The ions in our experiment will be laser-cooled, which would be disturbed by a magnetic field due to Zeeman splitting. Ergo, we chose the latter method since we would otherwise need an unnecessarily complicated laser apparatus. Our job is to make a computer simulation of an ion crystal with electrons inside a Paul trap, optimizing its parameters to attain the lowest possible temperature of electrons, hopefully reaching the electron delocalization over multiple ions in the CC.