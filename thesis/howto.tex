\chapter{Using software}

We have performed our calculations on a cluster with 32GiB of RAM, the 24-core processor 12th generation Intel Core i9-12900KF. The program is not optimized for minimal memory usage, and it is possible that some procedures will not run successfully on the less powerful machines.

\section{Python script}
\label{sec:code}
We do not provide an entire architecture of our program since it awaits an extensive reconstruction in the near future. In this appendix, we explain how to use the basic functionalities of the developed python script. Of course, it is best to follow actualized instructions on the \href{https://github.com/rendeka/Bachelor_thesis.git}{github\footnote{\href{https://github.com/rendeka/Bachelor_thesis.git}{$https://github.com/rendeka/Bachelor\_thesis.git$}}}. The program starts by running the \verb|main.py| file. This file has definitions of functions using many different modules. When running the \verb|main.py|, one should uncomment all the lines he/she wants to be executed. Fixed parameters can be adjusted in the module \verb|parameters.py|.

\begin{comment}

\xxx{This is an example code.}

\begin{Verbatim}
def StepVerlet(ODESystem, rv, t, dt, aCoulomb, mass, charge, trapParams):

    r, v = rv
    v, a = ODESystem(rv, t, aCoulomb, mass, charge, trapParams)
    
    r1 = r + v * dt + 0.5 * a * dt**2
    
    a1 = ODESystem(np.array([r1, v]), t, aCoulomb, mass, charge, trapParams)[1]    
    
    v1 = v + 0.5 * (a + a1) * dt
    t1 = t + dt
    
    rv1 = np.array([r1, v1])
    
    return rv1, t1   
\end{Verbatim}


\begin{listing}
\begin{lstlisting}
if __name__ == '__main__':
    
    prayForItToWork()
	    
\end{lstlisting}
\caption{Main program.}
\label{lst:main}
\end{listing}


\begin{listing}[H]
\begin{lstlisting}
	def ODESystemEffectiveDamping(rv, aCoulomb, mass, charge, trapParams):

    if (mass == electronMass):        
        a, q1, q2 = trapParams
        q1 = 0
    else:
        a, q1, q2 = trapParams * (electronMass / mass)
    
	    
    # unpacking position and velocity components
    r, v = rv
    x,y,z = r
    vx,vy,vz = v
    
    # defining the system of first order ODEs
    x1 = vx
    vx1 = aCoulomb[0] - x / 4 * (a + 2 * q1**2 / 2 * (f2 / f1)**2 
    	+ q2**2 / 2) - 2 * beta * vx
    
    y1 = vy
    vy1 = aCoulomb[1] - y / 4 * (a + 2 * q1**2 / 2 * (f2 / f1)**2 
    	+ q2**2 / 2) - 2 * beta * vy
    
    z1 = vz
    vz1 = aCoulomb[2] - z / 2 * (a + 2 * q1**2 / 2 * (f2 / f1)**2 
    	+ q2**2 / 2) - 2 * beta * vz
    
	# defining derivatives of position and velocity 
    r1 = np.array([x1, y1, z1])
    v1 = np.array([vx1, vy1, vz1])

    return np.array([r1, v1])	    
\end{lstlisting}
\caption{The system of ODEs used for simulating Coulomb crystal.}
\label{lst:crystal}
\end{listing} 

\end{comment}